% Kapitel 1 - Einleitung
\section{Introduction}
The focus of this Thesis describes the development and integration of a game-based approach to teach Python. The research directive is from a perspective of game elements and code submission. This is intended to help with the trend of educational games as they combine traditional game mechanics with programming components to aid learning in a game-based approach. The main issue is to ensure that the code submitted by the user is safe and secure, and how game elements can be designed and implemented for such a project.

The main purpose of this thesis is to outline the work of game development on an educational tool, which is designed to teach Python programming. The game is intended to be a fun and engaging way to learn Python programming to be played by beginners and experienced programmers alike.  It should encourage and motivate students to learn Python and coding in particular by letting players control their progress by using their individual programming skills. The way to make it fun and engaging is to use game elements such as levels, challenges, and rewards. The conceptualisation and design of such game elements will be explored in detail and how they should be implemented. Another exploration that will be made is the underlying mechanics of how code send by the player will be executed safely within a web app. The result of this project should be used to further develop the game-based learning project. The development also covers all work on the web application, focusing on the game elements and steps taken to mantain secure code execution. By reviewing existing approaches and implementing a prototype environment, this research evaluates the feasibility, scalability, and user-friendliness of such systems.

The sections discussed in this paper starts with the related work in Chapter 2, which will cover all related work that will be used in the project and an overview of the current state of the art in game-based learning . Chapter 3 will discuss the design and implementation of the game, including the game elements and the code execution environment. Chapter 4-6 will discuss the evaluation of the code, how to setup code execution and mantain its safety during.