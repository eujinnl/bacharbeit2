% Kapitel 9 - Fazit

\section{Conclusion}
Researching Python-based execution environments is a research directive that warrants more attention as the trend of educational games continues, such as Google’s Blockly \cite{google_blockly} and China’s Yuanfudao's Homework Help \cite{xiaoyuan_kousuan}.
The future steps of continuing to research Python-based execution environments are to look at key considerations such as the performance of both methods, what makes them secure, and possible alternative implementation methods. I also need to elaborate and clear up the process workflow such as detection/validation, cleaning/sanitization, submission, and evaluation phase. Lastly, I would also need to iterate on the design for modularity and scalability. The goal is to ensure a safe and secure working environment for what is needed to run such a service.

% \section{Fazit}
% Diese Arbeit befasste sich grundsätzlich mit der Frage, welche Anforderungen an ... gestellt werden.
% Der Lösungsvorschlag war...
% Benutzerstudien haben gezeigt, dass es einen signifikanten Unterschied... gibt. Diese Ergebnisse motivieren, um ...

% \subsection{Ausblick} 
% In dieser Arbeit haben wir uns mit ... beschäftigt. Durch zwei Benutzerstudien wurde festgestellt, dass sich die Aufteilung und Positionierung der Informationen innerhalb der Anzeige der beiden Gruppen aufgrund bla bla ändert. Das entwickelte Konzept ist zwar bla bla, müsste aber für eine Wiederholung der Studien bla bla zur Erhebung quantitativer Daten wie folgt angepasst werden:

% \begin{itemize}
% \item xxx
% \item yyy
% \item xzz
% \end{itemize}

% Des Weiteren wurde festgestellt, dass bla bla. Auch das müssten zukünftige Varianten besser berücksichtigen, z.\,B., indem sie bla bla. Für Studien sollte zudem eine neutrale(re) Umgebung gewählt werden. Somit sollte sichergestellt werden können, dass beispielsweise ein möglicher Bias der Marke des Fahrzeugs sich nicht auf das HMI-Konzept auswirkt...

% \subsection{Einschränkungen}
% Hinsichtlich Einschränkungen, die eine breite Anwendung der Ergebnisse verhindern, sind zwei getrennte Bereiche zu betrachten. Zum einen die Evaluation, welche sich nur auf den ersten Teil der Arbeit bezieht, zum anderen das ausgearbeitete Konzept für die Anzeige, das auf dem ersten Teil der Arbeit aufbaut. \\ [-2.5em]

% \paragraph{Evaluation} 
% Bla bla.

% \paragraph{Konzept}
% Bla bla. Das wurde etwa in beschrieben...

% %\input{chap9x} %chap9_futurework_limitations}