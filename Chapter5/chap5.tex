\section{Setup of code execution}
Python code submitted by the user should be executed to achieve some type of objective with constraints so that the objective can be achieved by the game/level designer's intended methods. Many unintended actions can be caused if user code is not run securely such as the modification of the intended objective, falsification of high scores, crashing the backend, or in the worst case, bypassing the game and changing system files and databases.

According to research, most methods to prevent malicious actions and limit potential damage by untrusted code can be classified into 2 categories: client-side setup and server-side setup. In this section, we will discuss the pros and cons of each method and how they can be implemented to ensure the security of the system.

\subsection{Client-Side setup}
By executing the code on client side with transpilation, the execution environment is ran client-side. This means that all potential damage is limited to the client only. If the code is malicious, their own page is broken and not the server.

To ensure security, the Python execution environment also needs to be restricted so that it can't load or modify other local data files. Sensitive attributes must be hidden outside the Python execution environment where the code will be ran. This means that any execution output should be only able to communicate 1 way as shown in figure \ref{fig:execution layout}, and the execution process should not be able to access anything beyond its scope, such as local files, backend and database.

\begin{figure}[H]
    \centering
    \includegraphics[width=0.8\textwidth]{images/execution_layout.png}
    \caption{Execution layout}
    \label{fig:execution layout}
\end{figure}


\subsubsection{Overview}
The client-side setup brings many benefits and disadvantages. If the client is disconnected from the server after loading the web app, the code can still be run locally on the machine as it has serverless program execution. Since the code is also run client-side, It can help reduce server load as no code is sent over and run locally on the client's machine. The client's machine is responsible for all code execution and therefore the execution speed is also reliant on it, getting rid of latency issues and making it more interactive. However, due to it running on the client's machine, there can also be resource constraints, slow execution, and crashing in the worst case. Any malicious intent will also just end up in the attacker breaking their own browser. Any high scores uploaded to the server can also be falsified due to the client-side execution and the server having no idea how the code was executed.

\subsection{Server-side setup}
A server-based setup is the more popular option for executing untrusted Python code and many such tools have been developed for it. This involves the code being sent from the client side to the backend which then can be executed, this usually involves the use of sandboxes. According to \cite{stephens2024sandbox}, Sandboxing is a technique for enforcing security policies on untrusted guest applications in a secure environment (i.e., “sandbox”) to eliminate risk to a host system. In this section, it's function is to securely execute untrusted application on host systems (CITE THIS). This execution happens usually in an isolated container or a seperate process with limited resources and various filters, such as system calls and network traffic etc.

Similar to the client-side setup, any execution output should only be sent without any influence to the backend database. Any execution processes should only happen within the isolated sandbox.

\subsubsection{Docker containers and Virtual Machines}
A Docker container is a lightweight, standalone, executable package of software that includes everything needed to run an application, not to be confused with Virtual machines (VMs). Similar to methods employed in some CTFs (Capture The Flag), and other online tools used to assess code(CITE THIS), Docker containers can also be used to sandbox and execute code isolatedly. The container can either be ran on a VM or on the same server as the backend.

Virtual Machines are also used to sandbox and execute code, but they are slower to boot up and can be expensive to run. They also require more resources than docker containers and can be harder to manage. However, they can be more secure than docker containers if they are isolated from the main server. These VMs can either be hosted on the same OS the server is ran on or hosted on the cloud with popular services such as Microsoft Azure and AWS EC2 instances.

The architecture of a server-side setup with docker containers would look like in figure \ref{fig:docker container setup}. It begins with the frontend sending code to the backend to run it through the container. The backend sends the code to the container along with any relevant information, such as the current level. This enables the container to run any pre defined logic and checks based on relevant information. The container then does validation and execution of the code. The container will be running a seperate server that can handle code execution and sends the standard output back to the backend, which then sends the results back to the client. 

\begin{figure}
    \centering
    \includegraphics[width=0.8\textwidth]{images/server_execution_layout.png}
    \caption{Docker container setup}
    \label{fig:docker container setup}
\end{figure}

To fire up and setup the container, a DockerFile which contains instructions on how to build the Docker image is used, this usually instructs what type of dependencies to install, setting environment variables and copying files etc. The DockerFile is then built into a Docker image using the 'docker build' command. The image is then used to create a docker container using the 'docker run' command, which is a shortcute of "docker create" and "docker start". A tool that can be used to speed up service running process is docker-compose. However, using a “dockercompose.yml” file, a container (or more than one) can be run with a single command (docker-compose up), in the background (using “-d” option) and the image can be built each time before container starts again.

To setup a virtual machine, a filesystem is needed, and a VM is needed to be spun up. Popular VM options include VirtualBox, VMWare, and Hyper-V. VMs can also run almost any operating software, the most common being some form of Linux distro such as Ubuntu. VM security is also treated very seriously and there have also been a lot more history and security research done about VMs compared to Docker containers, such as seccomp which was initially developed as a way to restrict system calls for Linux kernels. VMs also have the advantage of customization and OS configuration to install OS specific packages such as CodeJail to help aid the execution of untrusted code.

To gurantee absolute isolation and safety, both VMs and Docker containers can be used to construct the sandbox execution environment by hosting the Docker container on a cloud hosted VM that is seperate from the main backend.

\subsubsection{Overview}
Similar to client-side setup, potential benefits and disadvantages depend on how the system is designed. If one were to use docker containers to contain code execution, there would be a significantly increased server load, as virtual machines need to be booted up and served for each connection, as compared to executing code in a separate process in the server. However, docker containers have the advantage of limiting potential damage to only the OS it is hosted on. This setup also allows the server to log, monitor, rollbacks, and cache results from user code.

Another potential addon to the server-side setup is to by applying client-side code enforcement before the code is sent to the backend. In this approach, the client-side environment is used to enforce initial constraints and perform preliminary checks on the user-submitted code. This can include syntax validation, basic security checks, and ensuring that the code adheres to predefined rules. By doing this on the client side, we can quickly filter out obviously malicious or incorrect code before it reaches the server and reject it before it is sent, reducing the server load and improving responsiveness.

\subsection{Conclusion}
Both methods have their own pros and cons, and the choice of which method to use depends on the use case and the resources available. The client-side setup is more suitable for offline functionality and reducing server load, while the server-side setup is more suitable for monitoring, logging, and security.

