\section{Methods to prevents malicious Code Execution}
After the code is submitted, it will be ran in the sandboxed or virtualised environment that was setup in the previous section. As it is sandboxed, any code that is executed should be safe and limited in damage, even if the user inputs a HTML tag like <script>, it should only affect their version of the browser. However there are more prevention methods we should take. This section will discuss the methods taken to prevent untrusted code from executing and causing harm.

As discussed in previous subsection \ref{section:code_injection}, CIAOs taint the expected output program. The expected output program are fully evaluated values that we extract after code execution, such as a high score or tracked python variables that we can get from the execution environment's scope. If the expected output behaviour is not consistent with the expected output, the code is considered malicious. Here is a list of intended outputs from designed levels:
\begin{itemize}
    \item The intended output is a function that has to pass unit tests.
    \item The intended outputs is a tracked variable that has to be of a certrain type such as string, integer, or float.
    \item The intended output is a flag that has been hidden in the file system of the sandboxed environment.
    \item The intended output is a class that fits all abstract purposes.
    \item The intended output is if a keyword has been used and the result is a modified list/boolean/string.
\end{itemize}
Unintended outputs may, but are not limited to, the following:
\begin{itemize}
    \item The code is an infinite loop.
    \item The code is a database injection.
    \item The code is a privilege escalation.
    \item The code is a fork bomb.
\end{itemize}


% Code execution was initially implemented with a naive approach, as show in Listing \ref{lst:initial_approach}

\subsection{Preventing database injection}
Once user code is submitted and executed, only intended outputs from designed levels shouls be handled. This means that any communication between the execution environment and the database should be heavily controlled and monitored by the backend. Static analysis should be implemented on the code output, this means that the output is checked on the backend and should follow a predetermined ruleset with regular language. A few examples are as follows:
\newline
\begin{tabular}{|p{5cm}|p{11cm}|}
    \hline
    Intended output of player action & static analysis method\\
    \hline
    Player's name has to be a string & string is limited in length and cannot use any special escape characters\\
    \hline
    Player's avatar creation variables have to follow a color code & regex is used to check if it starts with "\#" and is only limited in length\\
    \hline
\end{tabular} 

Anything that has to be passed or updated with the backend should always include sanitation. Parameterized queries should also be used instead of directly concatenating user input into queries, most of these can be prevented by using an Object-Document Mapping(ODM) library like PyMongo. 

The ODM needs to be setup with the necessary security features such as a connection pool, this means that the database connection is reused instead of creating a new connection for every request. This is important as creating a new connection for every request can lead to a denial of service attack. As MongoDB is a NoSQL database, it is functionally different from SQL databases, it is schema-less and uses JSON-like documents. Collections schemas need to be setup to prevent injection attacks. This can be done with Basemodels from pydantic, this forces the data to be structured in python before being sent to the database. Regex methods should also be used to modify raw user input for safety. The lack of needing to write queries for data insertion is also a huge advantage as statements are already prepared with the ODM.

\subsection{Preventing resource exhaustion} 
When allowing user code to be ran, be it client or server side, it is important to prevent resource exhaustion. This can be done by setting a time limit on the code execution.

\subsubsection{Client-side resource exhaustion (Pyodide Implementation)}
Initially, code execution can be ran and limited with a timer using the signal modul in python to prevent infinite loops and resource exhaustion. This would have worked in native python as its interrupt system is based on preemptive multitasking, but Web Assembly has no support for preemptive multitasking. Pyodide is single threaded, so the signal module would be ignored anyways as pyodide will be stuck in a resource lock. Therefore, interrupting execution in Pyodide must be achieved via a different mechanism which takes some effort to set up.

According to the pyodide documentation, the best way to interrupt execution is to run code execution in a dedicated web worker. To setup an interruption mechanism with a timeout, a sharedarraybuffer can be used to communicate between the main thread and the web worker. Shared array buffers are a way to share memory between the main thread and the web worker, but requires special security headers in chrome browsers. The special headers implemented are  "Cross-Origin-Opener-Policy":"same-origin" and "Cross-Origin-Embedder-Policy":"require-corp". This ensures that all the resources are loaded from the same origin and that the web worker is running in a secure context to defend against attacks.

Pyodide is now ran in a web worker instead of the main thread, this has the advantage of not blocking the main thread. If a simple infinite loop like "While True: " is ran in the main thread, the browser will freeze and the user will not be able to interact with the page and has to restart the page while losing all progress. Running the code in a web worker will prevent this from happening as the main thread will still be able to interact with the page while returning an error message after the timeout if the code has not finished executing.

The interruption system puts together all these concepts by setting a sharedarraybuffer as the interrupt buffer (CITE THE DOCS). Before code is executed, the interrupt buffer is set back to default. After 3 seconds, this interrupt buffer is changed to interrupt the code execution. The web worker can then check the sharedarraybuffer for the changed signal and stop execution if the buffer is changed. This method is more complex than using the signal module in python, but it is the only way to interrupt execution in Pyodide for now until Pyodide supports threading. 

\subsubsection{Server-side resource exhaustion}
As mentioned previously, python modules can be used as an approach to prevent server-side resource exhaustion in the execution environment. Various modules like signal, and subprocess are used to run the code and kill it if it exceeds the time limit.
\begin{lstlisting}[language=Python, caption=Server-side resource exhaustion prevention with python module signal, label=lst:server_side_resource_exhaustion]
import signal
import sys

# Define the timeout handler
def timeout_handler(signum, frame):
    print("Time's up! Interrupting the infinite loop.")
    sys.exit(0)  # Terminate the script

# Register the signal handler for SIGALRM
signal.signal(signal.SIGALRM, timeout_handler)

# Set an alarm for 3 seconds
signal.alarm(3)

print("The script is running. It will automatically stop in 3 seconds.")

# Simulate an infinite loop
try:
    while True:
        print("Running...")
except SystemExit:
    print("Exiting the script.")
finally:
    # Cancel the alarm (cleanup, though not strictly necessary here)
    signal.alarm(0)
\end{lstlisting}

Other than using python modules, Docker containers can also be used to limit resources such as CPU and memory. This can be done by setting the --memory and --cpus flags when creating the container.
\begin{verbatim}
    docker run --memory="256m" --cpus="0.5" my-container
\end{verbatim}
This will limit the amount of memory and CPU the container can use. If a container is crashed, it will not affect the VM it is on as the container is isolated from the host machine. However, the user will not get a response if the container crashes. This lack of feedback will leave users unaware that their code execution has failed.

This is why termination of the container should be handled gracefully, the container should be stopped and removed if it crashes while redirecting the user to an error page or the code execution to a different container. This can be done with Kubernetes, a container orchestration tool that can manage the lifecycle of containers. Kubernetes can be used to automatically restart containers if they crash, and redirect the user to a different container if the container crashes.

\subsection{Code validation and sanitization}
The topic of code sanitation and validation have already been briefly mentioned, but it is important to reiterate the steps taken. 





% Code validation checks if the input meets a set of criteria (such as a string contains no standalone single quotation marks) and Sanitization modifies the input to ensure that it is valid (such as doubling single quotes)

% Code evaluation???




% The evaluation controller also analyzes the code for potential threats, inefficiencies, or policy violations. The validation controller should perform linting and syntax checking. It should also implement static code checking to avoid infinite loops and dangerous built-ins such as access to the OS. After that, a test runner should try to execute code but with limited time, memory, or CPU constraints.



% talk about eval and exec, how they are dangerous but unavoidable
% There are various problems with this, such as infinite loops


% talk about python being duck dynamic typed

% The database also needs to have a privilege separation, this means that the database user that the application uses should only have the necessary permissions to perform the necessary operations.





% Here is a list of intended outputs from designed levels:
% \begin{itemize}
%     \item The intended output is a function that has to pass unit tests.
%     \item The intended outputs is a tracked variable that has to be of a certrain type such as string, integer, or float.
%     \item The intended output is a flag that has been hidden in the file system of the sandboxed environment.
%     \item The intended output is a class that fits all abstract purposes.
%     \item The intended output is if a keyword has been used and the result is a modified list/boolean/string.
% \end{itemize}


% \subsection{Code Evaluation, Validation, and Sanitization}
% Code validation should be run before the code is executed, the code shall be parsed as a string to an evaluation controller to ensure the submitted code complies with the application's security policies and usage constraints. The evaluation controller also analyzes the code for potential threats, inefficiencies, or policy violations.
% The validation controller should perform linting and syntax checking. It should also implement static code checking to avoid infinite loops and dangerous built-ins such as access to the OS. After that, a test runner should try to execute code but with limited time, memory, or CPU constraints.






% \subsection{Server-side execution}
% Statically typed 

% \subsection{Client-side execution (Pyodide Execution)}
% This was the method that was explored with the most to figure out if there were any potential security risks. The Pyodide execution environment was already sandboxed out of the box. So even if there were any malicious code trying to destroy the environment, it would only affect the user's browser. Furthermore, pyodide

% tmp home dev proc lib posix

% There are also other attempts to sandbox Python code within the same execution environments. There are also tools that try to remedy this problem, RestrictedPython is a tool that defines a subset of Python language to provide input into our python environment. It gives strict bytecode access to the user code and restricts what is allowed to do but requires intimate knowledge of how Python works. 


% \subsection{Namespace Restrictions}
% If for example, dictionaries should not be used, the keywords and methods such as {} or dict.fromkeys() can be restricted as the user’s code is run in a local namespace. This can be particularly useful in preventing “import” and removing access to parts of the runtime. 
% However, this isn’t exactly foolproof as Python doesn't have strict private members like other languages such as Java or C++, allows one to walk up a class tree or access private variables such as “\_\_builtins\_\_” \cite{stackexchange_untrusted_python}. However, there is a promising project called “Restricted Python” \cite{restricted_python} currently under development that might prevent these cases in the future. 





% \section{Caution against code execution}
% After the code is submitted, it will be ran in the sandboxed or virtualised environment that was setup in the previous section. As it is sandboxed, any code that is executed