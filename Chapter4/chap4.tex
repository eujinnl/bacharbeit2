\section{Background on Code transpilation}
Source-to-source compilers, also known as transcompiler, transpiler or cross-compilers is a set of tools that take the source code developed in a high-level programming language (source language), and after a transpilation process, it generates a translated source code written in another programming language (target language), that is syntactically and approximately equivalent, the original source code can now be compiled and ran in the target language. 

\begin{figure}[H]
    \centering
    \includegraphics[width=0.4\textwidth]{images/transpiler_working.png}
    \caption{Principles of working for a transpiler}
    \label{fig:transpiler_working}
\end{figure}

\begin{lstlisting}[language=Python, caption=Example of python user code]
# Python Program to find the area of triangle

a = 5
b = 6
c = 7

# calculate the semi-perimeter
s = (a + b + c) / 2

# calculate the area
area = (s*(s-a)*(s-b)*(s-c)) ** 0.5
print('The area of the triangle is %0.2f' %area)

\end{lstlisting}

\begin{lstlisting}[language=JavaScript, caption=Example of a transpiled code]
// JavaScript Program to find the area of triangle

let a = 5;
let b = 6;
let c = 7;

// calculate the semi-perimeter
let s = (a + b + c) / 2;

// calculate the area
let area = Math.sqrt(s * (s - a) * (s - b) * (s - c));
console.log(`The area of the triangle is ${area.toFixed(2)}`);
\end{lstlisting}

We need to look into transpilation due to 2 major concerns, one being the fact that most browsers do not natively allow python code to be ran and secondly if our backend servers are not able to execute python code. In scenarios like that, we need to transpile our python user code to some other compatible language that can execute the user code in a different language and return the results back in Python.

Puder et al.[4], says that cross-compilation itself provides only the tools to translate from one language to another, but to make it usable, it is necessary to offer a library to provide the building blocks and there currently exists no such libraries to transpile the entire python ecosystem, only parts of it. Most of these transpilers consistently have issues with blocking synchronous user input. This introduces lots of bugs and causes crashing the user's browser, which was a huge problem during testing and implementation as it was also hard to debug. This section discusses the different transpiler tools available that would allow us to execute code.

\subsection{WebAssembly-based transcription tools}
WebAssembly is a low level, byte code instruction set that is run directly from most browsers, it targets the problem of safe, fast, portable low-level code on the Web (CITE). It allows execution of code at near-native speed, and is a portable target for compilers from any language. By using a Python-to-WebAssembly transpiler tool, we can run the code in the browser without the need for a server. After transpilation, user written Python can then be interpreted and compiled to byte code and executed directly.

An option that was under consideration to use for the project was PyodideU, as it allows synchronous user input, allowing blockable Python in the browser to be ran. It has many good features such as a Line by line debugger with access to local variable scope, Synchronous I/O and Graphics. Sadly, it lacks documentation as of the time of writing.

Another alternative was pyscript, as it was also a way to enable Python to run in the browser via webAssembly. Pyscript also allows python to be ran in the browser and is primarily used for web development.

Pyodide, a port of CPython to WebAssembly, was used as it is easy to implement was able to be run client-side and has a large, active community. Pyodide's main focus is to allow users to run Python code directly and also includes many general-purpose Python libraries, such as NumPy, Pandas, and regex.

\subsection{JavaScript-based transcription tools}
Javascript is a core technology of the web and is supported by all browsers. Like WebAssembly, it is also able to run client-side and is able to run user code and faces many of its challenges. Javascript is also a language that is used in backend frameworks like Express.js and next.js.

% rewrite this to illustrate each tool and not just glance over them
There are a few Python to JavaScript transpiler tools available, such as Brython, skulpt, filbert and transcrypt. Brython compiles the Python standard library into JavaScript on the context of a website browser, whereas Transcrypt uses a different approach and precompiles Python to JavaScript. Skulpt is a javascript implementation of python syntax by use of a foreign function API. Filbert is a javascript based Python parser that is largely abandoned with the last commit being 6 years ago. It has a very interesting approach to parsing python code to javascript and could be used for debugging user code or live programming debugging scenarios. The discussed transpilers are alright for teaching the fundamentals but lack the ability to run other modules and libraries that are not included in the standard library such as numpy, pandas and even in some cases the basic math module.