
%----------------------------------------------------------------------------------
%----Präambel/Preamble-------------------------------------------------------------
%----------------------------------------------------------------------------------

\documentclass[	a4paper,
				11pt,
				DIV=11,
				bigheadings,
				idxtotoc,
				listof=totoc,	
				bibtotoc,		
				halfparskip,
				cleardoubleempty,
				oneside,
				openright]{scrartcl}
%----------------------------------------------------------------------------------

\usepackage[english]{babel}
\usepackage[T1]{fontenc}
\usepackage[utf8]{inputenc}
\usepackage{subfig}

\usepackage[OT1]{fontenc}
\renewcommand*\familydefault{\sfdefault}


\usepackage{graphicx}	

\usepackage[labelfont=bf]{caption}
					
\usepackage{float}
\usepackage{wrapfig}
%\usepackage{subfigure}

\usepackage{geometry}						% Für newgeometry in Titelseite
\geometry{a4paper,left=30mm,right=20mm}

\usepackage{blindtext}
\usepackage{layout}

\PassOptionsToPackage{hyphens}{url}		
\usepackage[pdfborder={0 0 0},
			colorlinks=true, 
			linkcolor=black,
			citecolor=red,
			]{hyperref}
			
\usepackage{natbib}%numbers 				

\usepackage{pdfpages}

\usepackage{color}
\usepackage{xcolor}

\usepackage{setspace}

\usepackage{longtable}
\usepackage{multirow}
\usepackage{colortbl}

%----------------------------------------------------------------------------------
%----Kopfzeile---------------------------------------------------------------------
%----------------------------------------------------------------------------------

\usepackage{scrlayer-scrpage} 							% Aufruf KOMA-Skript für Kopfzeilen

\pagestyle{scrheadings}							% Definition der Eigenen Headerformatierung
\clearscrheadfoot 								% alle Standard-Werte und Formatierungen raus
%\automark[chapter]{section}						% Kapitel und Section als Inhalt der Variablen leftmark und rightmark
\ohead{\pagemark}								% Seitenzahl auf äußerem Rand
\ihead{\ifthispageodd{\leftmark}{\rightmark}} 	% Innere Überschrift mit Kapitel bei linker Seite und Section bei rechter Seite -> geht nur in Verbindung mit
												% zweiseitigem Text wirklich sinnvoll
\setheadsepline{0.4pt} 							% Trennlinie Fußzeile und Textkörper
\setkomafont{pagehead}{\scshape}				% Schriftart in Kopfzeile, \scshape = Kapitelchen
%----Fußzeile----------------------------------------------------------------------
\setfootsepline{0.4pt} 							% Trennlinie Fußzeile und Textkörper
\setkomafont{pagefoot}{\scshape}				% Schriftart in Fußfzeile, \scshape = Kapitelchen
\ifoot{\footnotesize{Leong Eu Jinn}}
\ofoot{\footnotesize{Bachelor thesis}}		
%----------------------------------------------------------------------------------

\defpagestyle{myPageStyle}{
	(0pt ,0pt)
	{\hfill\pagemark} {\hfill\pagemark} {\hfill\pagemark}
	(0pt ,0pt)	
}{
{
	(\textwidth ,0.4pt)
	\footnotesize{Leong Eu Jinn} \hfill \footnotesize{Bachelor thesis}} 
	{\footnotesize{Leong Eu Jinn} \hfill \footnotesize{Bachelor thesis}} 
	{\footnotesize{Leong Eu Jinn} \hfill \footnotesize{Bachelor thesis}}
	(0pt ,0pt)
}


%----Farbdefinition--THI-blau------------------------------------------------------
\definecolor{haw_mag}{rgb}{0,0.112,0.47}
\addtokomafont{section}{\color{haw_mag} \rmfamily \scshape} 
\addtokomafont{subsection}{\color{haw_mag} \rmfamily}
\addtokomafont{subsubsection}{\color{haw_mag} \rmfamily}
\addtokomafont{paragraph}{\color{haw_mag} \rmfamily}
\addtokomafont{subparagraph}{\rmfamily}
%----------------------------------------------------------------------------------

\definecolor{tab_2}{RGB}{230,230,230}
\definecolor{tab_1}{RGB}{85,128,214}

%------Längenanpassung-------------------------------------------------------------
\setlength{\headsep}{10mm}						% Textabstand zur Kopfzeile
\setlength{\footskip}{15mm}						% Abstand zur Fußzeile
\setlength{\textheight}{235mm}					% Texthöhe
%----------------------------------------------------------------------------------


%----Glossar-----------------------------------------------------------------------
\usepackage[toc, acronym]{glossaries} 			
\makeglossaries	

%----------------------------------------------------------------------------------
%----Glossar-----------------------------------------------------------------------
\usepackage[intoc]{nomencl} 
\makenomenclature
%----------------------------------------------------------------------------------



\includeonly{	
	titlepage,							
	affidavit,
	acknowledgments,
	%abstractDE,
	abstractEN,
	confidentialityClause,
	glossary,
	mainpart,
	%outlook,
	%fazit,
	nomenclature,
	appendices
	}
	
			
%-------------------------------------------------------------------------------------------------------------------------------------------------------------
%----------------DOKUMENT-BEGINN---------------------------------------------------
%-------------------------------------------------------------------------------------------------------------------------------------------------------------

\begin{document}

	%\shorthandoff{"}						% Vermeidung von ungewollten Ligaturen/Avoid unwanted ligatures
	
	%----Vermeidung von Hurenkindern und Schusterjungen---------------------
	\widowpenalty=10000
	\clubpenalty=10000
	\displaywidowpenalty=10000	
	%-----------------------------------------------------------------------

	%Titelseite/title page	
	%----------Titelseite-------------------------------------------------------------

\newgeometry{textheight=0.9\paperheight, textwidth=0.76\paperwidth, left=30mm, right=20mm}

\begin{titlepage}	
	%----THI+(x-company)-logo--------------------------------------------------------
	\begin{figure}[!h]
		\centering
		    %---- Add Cooperation here and add scaling if necessary 
			\includegraphics[width={0.2\textwidth}]{images/ai-motion.png}	
			\hfill
			\includegraphics[width={0.1\textwidth}]{images/cvims_logo_dark.png}	
			\hfill
			\includegraphics[width={0.2\textwidth}]{images/thiRGB.jpg}	
		\end{figure}																			
	%------------------------------------------------------------------------------
	
	\begin{center}
		\hrulefill 
	\end{center}
	
	
	\begin{center}	
		\vspace{1cm}
		\huge\textbf{Technische Hochschule Ingolstadt}\\[1em]
		\Large \textbf{Faculty of Computer Science}\\ 
		\normalsize
		Computer Vision for Intelligent Mobility Systems \\ 
		Study Program Computer Science and Artificial Intelligence \\ [2.5em]
	\end{center}


	\begin{center}	
		\vspace{1cm}
		\Large \textbf{Bachelor Thesis}\\ 
		\normalsize
		for the degree of \\ 
		Bachelor of Science (B. Sc.) \\ [3.5em]
		\huge\textbf{A study on integrating a game-based approach to teach Python programming from game elements and code submission perspective}	 \\ [3.5em]
	\end{center}



	
	\begin{center}
		\vspace{1cm}
		\hspace{1cm}
		\begin{tabular}{r@{:}ll}
			\textbf{Name and surname} & & \textbf{Leong Eu Jinn}	\\ [3em]
			
			\textbf{Issued on}	& & October 1, 2024	\\ [1em] % issuing date
			\textbf{Submitted on}	& & March 1, 2020	\\ [3em] %date of hand in
			
			\textbf{First Examiner} &	& Prof. Dr. Torsten Schön	\\ [1em]
			\textbf{Second Examiner} 	& & Prof. Dr. Patrick Cato	\\[3em]
			
			% % \textbf{Faculty advisor} &	& PhD student \\ [1em] %if applicable 
			% \textbf{Supervisor at COMPANY} &	& Mr. The Expert \\ %if applicable
		\end{tabular}
	\end{center}
	
\end{titlepage}

\restoregeometry					% include erzeugt immer eine neue Seite bei jedem Einbinden
	\cleardoublepage						% include always creates a new page
	
	\pagenumbering{Roman} 			% Römische Nummerierung der Kapitel/roman page numbering
	
	%Erklärung
	\thispagestyle{myPageStyle}
	%----------Eidesstattliche Erklärung/Affidavit--------------------------------------

\addsec{Affidavit}  % Erklärung/
I declare that I have authored this thesis independently, that I have not used other than the declared sources/resources, that I have not presented it elsewhere for examination purposes, and that I have explicitly indicated all material which has been quoted either literally or by consent from the sources used. I have marked verbatim and indirect quotations as such.	\\[2em]
	
Ingolstadt, \rule{0.3\textwidth}{0.4pt}	\\ [1.5cm]
	%\textcolor{white}{.}\qquad\qquad\qquad\qquad\quad \small (Datum) \\ [1.5cm]
	
%(Unterschrift) \\
Leong Eu Jinn


	\cleardoublepage
	
	%Danksagung
	\thispagestyle{myPageStyle}
	%----------Danksagung/Acknowledgments--------------------------------------------------------------
\addsec{Acknowledgments} % Danksagung/

This is the sentimental part where you get to thank all the persons who were a part of your 
thesis journey in one or the other way!
		
Leong Eu Jinn\\
Ingolstadt, Germany\\
November xx 2024
	\cleardoublepage
	
	%Kurfassung/Abstract German (only for thesis written in German)
	%\thispagestyle{myPageStyle}
	%%----------Kurzfassung DEUTSCH----------------------------------------------------------------

%\addsec{Kurzfassung}
%Deutschsprachige Kurzfassung...
	%\cleardoublepage
	
	%Kurzfassung/Abstract Englisch (for every thesis)
	\thispagestyle{myPageStyle}
	%----------Zusammenfassung Englisch/Abstract----------------------------------------------------------------
\addsec{Abstract}
The integration of game-based learning in education has revolutionized traditional learning methods, making them more interactive and engaging (\cite{prensky2003digital},\cite{tobias2014game}). Python, renowned for its simplicity and versatility, is increasingly used to teach programming, particularly to beginners. This thesis explores the development of implementing Python-based programming games from a gameplay and user experience perspective.
\\\\
The study addresses some key challenges of ensuring the secure execution of Python code through an Angular web app, maintaining performance in a web-based context, and designing user-friendly interfaces that foster an interactive learning experience. By reviewing existing approaches and implementing a prototype environment, this research evaluates the feasibility, scalability, and user-friendliness of such systems.
\\\\
This thesis contributes to the growing field of educational technology, offering insights into developing innovative tools that make programming education more accessible and engaging for learners worldwide. The development also covers the work on the web application to make the new concepts as highly adaptive as possible
%------------------------------------------------------------------------------------------------------------

	\cleardoublepage
	
	%Sperrvermerk/Confidentiality clause (if any)
	\thispagestyle{myPageStyle}
	%----------Sperrvermerk/Confidentiality clause------------------------------------------------------------


\addsec{Confidentiality clause} % Sperrvermerk/

Optional.\\ 
	
Ingolstadt, \rule{0.3\textwidth}{0.4pt}	\\
\textcolor{white}{.}\qquad\qquad\qquad\qquad\quad \small (Date) \\ [1.3cm]
	
(Signature) \\
Firstname Lastname
	\cleardoublepage
	
	
\nomenclature{$\sigma$}{Sigmoid function}
\nomenclature{$w$}{Neural network weights}
\nomenclature{$e$}{Euler's number}









		
	\printnomenclature
	\cleardoublepage	
	
	
%----------Glossar/Glossary-------------------------------------------------------------
% Anzeige erst auf Tools>Glossary bei jeder Änderung!!
\newglossaryentry{abs}{name={abs},description={Absolute operation}}  



\newacronym{d}{D}{Dimensions}
\newacronym{rgb}{RGB}{Red Green Blue channels}




	
	\printglossaries	
	\glsaddallunused
	\cleardoublepage

	
	%Abbildungsverzeichnis/List of figures
	\thispagestyle{myPageStyle}
	\renewcommand*\listfigurename{List of figures} % Remove for German thesis
	\listoffigures
	\cleardoublepage
	
	%Tabellenverzeichnis/List of tables
	\thispagestyle{myPageStyle}
	\renewcommand*\listtablename{List of tables} % Remove for German thesis
	\listoftables
	\cleardoublepage
	
	% Inhaltsverzeichnis
	\thispagestyle{myPageStyle}
	\renewcommand{\contentsname}{Table of contents} % Remove for German thesis
	\tableofcontents
	\cleardoublepage
	\singlespacing
	
%--------------------------------------------------------------------------------	
%------Ausarbeitung--------------------------------------------------------------
%--------------------------------------------------------------------------------

	\pagenumbering{arabic} 						% Arabische Nummerierung der Kapitel/Arabic page numbering
	%----------Hauptteil/Main part of the thesis-----------------------------------------------------------
\thispagestyle{myPageStyle}
% Kapitel 1 - Einleitung
\section{Introduction}
The focus of this Thesis describes the development and integration of a game-based approach to teach Python. The research directive is from a perspective of game elements and code submission. This is intended to help with the trend of educational games as they combine traditional game mechanics with programming components to aid learning in a game-based approach. The main issue is to ensure that the code submitted by the user is safe and secure, and how game elements can be designed and implemented for such a project.

The main purpose of this thesis is to outline the work of game development on an educational tool, which is designed to teach Python programming. The game is intended to be a fun and engaging way to learn Python programming to be played by beginners and experienced programmers alike.  It should encourage and motivate students to learn Python and coding in particular by letting players control their progress by using their individual programming skills. The way to make it fun and engaging is to use game elements such as levels, challenges, and rewards. The conceptualisation and design of such game elements will be explored in detail and how they should be implemented. Another exploration that will be made is the underlying mechanics of how code send by the player will be executed safely within a web app. The result of this project should be used to further develop the game-based learning project. The development also covers all work on the web application, focusing on the game elements and steps taken to mantain secure code execution. By reviewing existing approaches and implementing a prototype environment, this research evaluates the feasibility, scalability, and user-friendliness of such systems.

The sections discussed in this paper starts with the related work in Chapter 2, which will cover all related work that will be used in the project and an overview of the current state of the art in game-based learning . Chapter 3 will discuss the design and implementation of the game, including the game elements and the code execution environment. Chapter 4-6 will discuss the evaluation of the code, how to setup code execution and mantain its safety during.			% 1. Einleitung/Introduction and problem statement
\newpage

\thispagestyle{myPageStyle}
% Kapitel 2 - Related Work / Literaturanalyse
\section{Related Work}
This section discusses about game development of educational games, gamification and other tools focused on teaching programming skills.
%  but with different game elements and didactic approaches.

\subsection{Gamification and Game-Based Learning}
\cite{10.1145/2181037.2181040} defines gamification as "the use of game design elements in non-game contexts", this increases the engagement of the learner   (CITE HERE). Figure 1 classifies gamification and differentiates it to similar areas. There is a large number of game mechanics that can be added in terms of gamification (Brull and Finlayson, 2016; S. Kim et al., 2018):
\begin{itemize}
    \item Points
    \item Badges
    \item levels
    \item Leaderboards
    \item Progression Bars
    \item Certificates
    \item Story
    \item Avatar (selection and customization)
\end{itemize}

Examples of gamification can be seen in apps like Kahoot! and Duolingo. They are platforms in which lessons and quizes are given out, these are traditional questions that are usually given out in a learning or exam setting, but with now extra unlockables and collectibles. 

\begin{figure}[H]
    \centering
    \includegraphics[width=0.5\linewidth]{images/dimensions.png}
    \caption{The dimensions of gaming/playing and whole/parts (After \cite{10.1145/2181037.2181040})}
\end{figure}

In order to distinguish gamification from game-based learning, gamification just introduces gamelike elements (elements or mechanics) into a non-gaming setting. Game-based learning, however, is a type of game play that has defined learning outcomes (Becker, 2021). With a specific learning goal in mind, a learning task is redesigned to make learning more interesting and more effective. This involves the use of serious games and elements of gamification in the learning process, seen as a tool of game-based learning. 

\subsection{Game Frameworks}
To date, educational game development teams have utilized a diverse mix of game design and instructional design methodologies to help realize their designs, but often without a unifying framework to bring these diverse perspectives together. An iterative approach to designing educational games is Winn's (2008) Game Design Play and Experience Framework, which is a modificaiton of the Mechanics, Dynamics, and Aesthetics (MDA) Framework as it does not address aspects outside of Entertainment. According to Figure 1.2, the designer of an educational game would usually have to take into account 4 different layers[17]. Learning, Storytelling, gameplay and user experience. This thesis focuses on the 3rd and 4th layer of the framework, gameplay and user experience.

\begin{figure}[H]
    \centering
    \includegraphics[width=0.5\linewidth]{images/dpe framework.png}
    \caption{Design, play and experience framework}
\end{figure}


The gameplay layer most closely resembles another framework. The gameplay layer defines what the player does in the game and is broken down into mechanics, dynamics and effects. The mechanics are the rules that define the operation of the game world, what the player can do, the challenges the player will face, and the player’s goals. The dynamics are the resulting behavior when the rules are instantiated over time with the influence of the
player’s interactions. The resulting experiences, or emotions derived in the player, are the affects. This is the rules and operations of the game world and the background processes of our game. For the purposes of the project, our focus on executing code is the core mechanic and dynamic of the game.

The user experience layer is designed so that user interfaces are made to access the entertaining gameplay (Saltzman, 2000, p. 256) and to create a vehicle to realize the desired serious outcomes. Good user interfaces are said to be transparent, that is, the player does not have to focus their attention on how to play the game (i.e., what button to press) but rather on the gameplay, storytelling, and learning experience.

The aspects of gameplay and user experience layer of the designer are key areas of game development concern that should also be implemented for educational games. The conceptualisation and implementation of such game elements/mechanics for the project will be explored in depth in later sections.

\subsection{Other educational tools in Computer Science}
There are also many other educational tools or games aimed at teaching programming that follow a game-based learning approach. Such games can usually be divided into three categories according to \cite{combefis2016learning}:
According to \cite{combefis2016learning}, there are mainly three categories of games in Computer Science:
\begin{itemize}
    \item \textbf{Coding}: The focus is to make users learn and train to code. Coding games require the learner to understand and to be able to write code to solve challenges. A central part is to understand the syntax behind a programming language.
    \item \textbf{Algorithmic thinking}: The focus is not on learning a particular programming language and relating concepts, but on learning concepts like algorithms and data structures. The system provides various problems that have to be solved in a technical way. This does not necessarily use programming languages, but can be done with other concepts.
    \item \textbf{Creating games}:  It is a kind of online programming learning platform that offers the possibility for the users to create their own games. On these platforms, the learner has to program a game, typically with a visual programming language. 
\end{itemize}

Other interesting key features of educational games and tools will be observed and analysed.

\subsubsection{Minecraft and Minecraft: Education Edition}
Minecraft Education is a game-based platform that inspires creative, inclusive learning through play (CITE THIS, ITS LITERALLY FROM THEIR WEBSITE IDK HOW). Of the 3 categories of games defined above, Minecraft Education fits all 3. Minecraft: Education Edition can teach the fundamentals of programming languages such as Python and Javascript by providing courses and classrooms to students and its very own code builder to teach syntax. Multiple studies, such as \cite{kuhn2018minecraft} and \cite{9803384} has also found that Minecraft has been found to boost creativity and allow users to create games, chatbots and even in some cases virtual computers.
\begin{figure}[H]
    \centering
    \includegraphics[width=0.5\linewidth]{images/blocks1_2.png}
    \caption{Minecraft Education: Code Builder}
\end{figure}

The key feature about Minecraft is its' sandbox and giving the user infinite choices and freedom to create whatever is desired.

\subsubsection{CodeCombat}
Launched in October 2013, CodeCombat is an open-browser-based game where players can learn coding with various programming languages (for example Python or JavaScript) and the fundamentals of computer science (CITE THIS). It is a 2D RPG where players control a character by programming it with code. There are goals set out for each and every level, teaching users fundamentals of programming and algorithms depending on the stage. Concepts are taught 1 by 1 progressively, where they start out with syntax, to conditions, loops and eventually algorithms or even making their own web-app. The tool also provides visual feedback and a very clean UI.

\begin{figure}[H]
    \centering
    \includegraphics[width=0.5\linewidth]{images/codecombat.png}
    \caption{CodeCombat Level}
\end{figure}
Ultimately, I think CodeCombat is a very well made game that should be the project's standard. It's adaptive to multiple programming languages and has a very clean looking UI.

\subsubsection{Duolingo}
Although Duolingo is not a game-based learning tool, many of its gamification elements should be observed and implemented in the project as they enhance user engagement and motivation (\cite{mekler2017towards}). Based on \cite{toda2019analysing}, Duolingo incorporates several key gamification elements:
\begin{itemize}
    \item \textbf{leaderboard}: Users can see how they rank compared to others based on their performance and progress, fostering \textbf{competition} and motivation.
    \item \textbf{Achievements}: Various achievements can be unlocked by completing specific tasks or reaching milestones, providing a sense of accomplishment (\textbf{Acknowledgement}).
    \item \textbf{Levels}: Users \textbf{progress} through different levels as they complete lessons, which helps to structure the learning process and provide clear goals.
    \item \textbf{Avatar Customization}: Users can customize their avatars (\textbf{Imposed Choice}), adding a personal touch and increasing engagement.
\end{itemize}

These elements can be learnt from and adapted to the project.
\begin{figure}[H]
    \begin{tabular}{cc}
    \subfloat[Avatar customization]{\includegraphics[width=2in]{images/duo_avatar.png}} &
    \subfloat[Level selector]{\includegraphics[width=2in]{images/duo_levels.jpg}} \\
    \subfloat[leaderboard]{\includegraphics[width=2in]{images/duo_leaderboard.png}}&
    \subfloat[Achievements]{\includegraphics[width=2in]{images/duo_achievements.png}}\\
    \end{tabular}
    \caption{Duolingo gamification elements}
\end{figure}			% 2. Literaturanalyse/Related work analysis
\newpage

\thispagestyle{myPageStyle}
\section{Implementation}
Game Mechanics function as basic systems of a game that governs their respective game elements (\cite{adams2012game}). All possible basic functions(represented by algorithms and data structures) and rules in the game are part of the mechanics. The section will discuss about planned game mechanics and their implementation.

\subsection{Web application}
The game will be a web application, this is because web applications are easily accessible and can be played on any device with a web browser. The game will be built using Angular as its frontend, a popular web application frontend framework and Flask as its backend, as it is a python lightweight backend compared to its alternatives.

As central storing database, MongoDB was used. This is because MongoDB is a NoSQL database, which is a good choice for storing JSON data and long strings of user code. This database's main purpose is to stores all relevant information regarding the game content and will contain possible analytics in the future. It's base entity is the "Users" entity as shown in Figure \ref{fig:users}. Upon the creation of a "Users" entity, all other relating entities with references should also be created.
\begin{figure}[h]
    \centering
    \includegraphics[width=0.5\linewidth]{images/user_object.png}
    \caption{Database model of the \textit{Users} entity}    
    \label{fig:users}
\end{figure}

\subsection{Level selection}
The game will have multiple levels, each level will have a different set of challenges. The player will have to complete each level to progress to the next. The levels will be designed to increase in difficulty as to teach the player different concepts of Python programming.

In the initial level selection, the player will be presented with a list of levels. Each level will have a title, a description, and a button to start the level. The player will be able to see the buttons of the levels they cannot access disabled and greyed out until they have completed the previous level. This progression of the player is kept track by the leaderboard, which will be discussed in the next section.

\subsection{Leaderboard}
According to \cite{https://doi.org/10.1111/jcal.13077}, different designs of leaderboards can maximise either performance, motivation or engagement. Lower ranking students should be anonymised to prevent demotivation, and an absolute real leaderboard which shows the points \& ranking of the students should also be in place to encourage competition. Figure \ref{fig:leaderboard} shows the final design of the leaderboard decided on, it only shows the top 3 players and the player's own ranking of each level. High scores for each level are based on arbitrary metrics, it can be time taken to complete or number of lines of code written, etc.
\begin{figure}[H]
    \centering
    \includegraphics[width=0.5\linewidth]{images/leaderboard.png}
    \caption{Leaderboard design}    
    \label{fig:leaderboard}
\end{figure}

The leaderboard will store each player's data in the form of a leaderboard entity in the database as shown in Figure \ref{fig:leaderboard_object}. Upon creation of the entity, all scores are set to 0 to show that the player has not completed any level. This way, the leaderboard functions as both a way to track the player's progress and as a way to store the player's highscore.
\begin{center} 
    \begin{figure}[H]
        \centering
        \includegraphics[width=0.5\linewidth]{images/leaderboard_object.png}
        \caption{Database model of the \textit{Leaderboard} entity}    
        \label{fig:leaderboard_object}
    \end{figure}
\end{center}
The backend will have an endpoint to handle the leaderboard, this endpoint will have basic CRUD functionality for the following basic use cases:
\\\\
% It will also handle data validation and edge cases in a following manner:
\begin{table}[H]
    \caption{Basic use cases for leaderboard endpoint}
    \begin{tabular}{|p{11cm}|p{5cm}|}
        \hline
        Use Case & Basic path\\
        \hline
        Player does not have a valid matching "User" entity & Return an error\\
        \hline
        Player has completed a level & Create/Update the player\\
        \hline
        Player wants to view the leaderboard & Read the database\\
        \hline
        Player recompletes a level with a lower highscore & Read the leaderboard \\
        \hline
        Player recompletes a level with a new highscore & Read and Update the leaderboard\\
        \hline
    \end{tabular}
\end{table}


\subsection{Achievements}

According to \cite{hamari2011framework}, achievements are composed mainly of a signifier, completion logic, and a reward. The Signifier element consists of the visible parts of the achievement. The foundational logic of an achievement defines the trigger (a player-invoked action or a system-invoked event), how many times it has to be triggered, under which conditions, and what pre-requirements exist. The Reward element defines the reward(s) a player acquires after unlocking the achievement; this can be as simple as a number going up. test2

\begin{wrapfigure}[16]{r}{0.5\textwidth}
    \centering
    \includegraphics[width=0.5\linewidth]{images/achievement_framework.png}
    \caption{Achievement Framework from \cite{hamari2011framework}}
    \label{fig:achievement_framework}
\end{wrapfigure}

The signifier agreed upon for the project is shown in Figure~\ref{fig:achievements} and will be displayed on the achievements tab. The completion logic for each achievement will differ, but most achievements are planned to be achievable within the same level. Hence, there is no condition that spans across levels, and there is no need to store pre-requirements in the database.

\begin{figure}[H]
    \captionsetup{justification=raggedright,singlelinecheck=false}
    \includegraphics[width=0.5\linewidth]{images/example_achievement.png}
    \caption{Example of an achievement}
    \label{fig:achievements}
\end{figure}
% \begin{wrapfigure}{l}{0.5\textwidth}
%     \centering
%     \includegraphics[width=0.5\linewidth]{images/example_achievement.png}
%     \caption{Example of an achievement}
%     \label{fig:achievements}
% \end{wrapfigure}

For basic achievements to be stored, an achievement database schema has to be designed to relate to the player schema. The achievements will be stored in the database as an entity, as shown in Figure~\ref{fig:achievements_object}. Upon creation of the entity, all entries of `time-obtained` are set to 0, indicating that the player has not completed any achievements.
\\\\
\begin{figure}[H]
    \centering
    \includegraphics[width=0.5\linewidth]{images/achievements_object.png}
    \caption{Database model of the \textit{Achievements} entity}
    \label{fig:achievements_object}
\end{figure}

Basic validation and verification also have to be implemented in the application layer to ensure no errors occur. The endpoint will have basic CRUD functionality for the following use cases:

\begin{table}[H]
    \centering
    \caption{Basic use cases for achievements endpoint}
    \label{table:achievements_usecases}
    \begin{tabular}{|p{11cm}|p{5cm}|}
        \hline
        \textbf{Use Case} & \textbf{Basic Path} \\
        \hline
        Player does not have a valid matching "User" entity & Return an error \\
        \hline
        Player has never had an achievement and triggers the completion logic of the achievement & Update/Create the player in the database \\
        \hline
        Player already has the achievement and retriggers the completion logic of the achievement & Do nothing \\
        \hline
        Player wants to view the achievements screen & Read from the database \\
        \hline
    \end{tabular}
\end{table}

\subsection{Character Creation}
According to \cite{anastasio26impact} and \cite{adams2013crash}, character creation is a significant part of the game as it can affect the player's experience. The character creation screen will be the first screen the player sees when they start the game. The player will be able to choose their in game name and appearance. These aspects are popular in games as they better allow the player to identify himself or herself with the game, which potentially leads to deeper engagement.

For characters to be stored, a character database schema has to be designed to relate with the player schema. The character will be stored in the database as an entity shown in Figure \ref{fig:character}. The basic implementation is that any submission of a character entity done by the user will always overwrite any current existing character entity in the database.
\begin{figure}[H]
    \centering
    \includegraphics{images/character_object.png}
    \caption{Database model of the \textit{Character} entity}
    \label{fig:character}
\end{figure}

\subsection{Main UI and game state implementation}
The UI plays a huge part of any game and was developed with PixiJS, A HTML5 creation engine that renders 2D graphics. There was no generic game engine used. The Main UI shows the player's code, the output of the code, and the main place where player inputs their code.

\begin{figure}[H]
    \centering
    \includegraphics[width=0.5\linewidth]{images/main_ui.png}
    \caption{Main UI of the game}    
    \label{fig:main_ui}
\end{figure}
For the UI to be reactive to any changes by the player, a game state manager was implemented. The game state manager is a singleton class that manages the state of the game and uses dependency injection. It is responsible for accepting code from the user and then sending it to a system that runs the code, receiving updates of achievements and certain variables that it needs to keep track of, and updating the UI accordingly. The game state manager is labelled as playerCodeService and can be shown in Figure \ref{fig:game_state}.
\begin{figure}[H]
    \centering
    \includegraphics[width=0.5\linewidth]{images/gamestate.png}
    \caption{Class Diagram of game state mamanger}    
    \label{fig:game_state}
\end{figure}

\subsection{Code Input(Game Controller)}
\begin{wrapfigure}{r}{0.5\textwidth}
    \centering
    \includegraphics[width=0.5\linewidth]{images/textbox.png}
    \caption{Simple basic text box (\cite{azatbekuly2023enhancing})}
    \label{fig:textbox}
\end{wrapfigure}
The game controller is the main interface between the game and the player, it is a system that accepts the player's code. The game controller will be responsible for checking basic errors in player's code and updating the game state and the UI after interacting with it. 

When building the project, the main options were a block based system as shown in Figure \ref{fig:minecraft_code_builder} or a text-based input system. Initially a plain textbox was initially used to input code as show in Figure \ref{fig:textbox}. 

To improve on this submission of code, we can use a code editor like professional integrated development environments (IDEs) due to their many features(\cite{kapoor2024analysis}). The code editor, Monaco was decided upon compared to other editors such as Ace and CodeMirror. Monaco also needed special configurations in the bundle system, and was tough to get it to cooperate. Features that are included and being worked on are:
\begin{itemize}
    \item \textbf{Syntax/Semantic Highlighting}: This displays text, especially source code, in different colors and fonts according to the category of terms and rules. This feature helps the user to understand the code better.
    \item \textbf{Code Formatting}: This is the process of organizing and structuring code according to predefined style rules. This involves automatically adjusting the indentation, spacing, and alignment of code elements when entering a new line to maintain a consistent appearance.
    \item \textbf{Autocomplete}: Automatically suggests possible completions for partially typed words or code construct.
    \item \textbf{Line Numbering}: Displays the line number of the code for quick referncing.
    \item \textbf{Code Folding}: Feature that allows the user to hide and display sections of code to make it easier to navigate and read.
    \item \textbf{Language specific features}: This needs to be improved upon to include language-specific features such as linting, definition provider, etc. Since Monaco did not come with python out of the box and custom modules had to be written to cover some basic features.
\end{itemize}
\begin{figure}[H]
    \centering
    \includegraphics[width=0.3\linewidth]{images/code_editor.png}
    \caption{Code editor of IDE, a significant improvement over a basic textbox}
    \label{fig:code_editor}
\end{figure}
The gamestate object can also control what appears on the code editor. Upon code submission, the next section will discuss the code execution system that will be used to run the code and check for errors.

% there is a reaction of the code that ran. This reaction of running code is usually known as standard output(stdout). This has to be printed out, including all results and any errors of the code. Other than just text printing with results of code submission, storytelling can take place here that is reactive upon successful results. 
		  % 3. Implementation, Technical Setting, Prototype
\newpage

\thispagestyle{myPageStyle}
\section{Setup of code execution}
Python code submitted by the user should be executed to achieve some type of objective with constraints so that the objective can be achieved by the game/level designer's intended methods. Many unintended actions can be caused if user code is not run securely such as the modification of the intended objective, falsification of high scores, crashing the backend, or in the worst case, bypassing the game and changing system files and databases.

According to this blog post by \cite{codecombat_aether} and some research, most methods to prevent malicious actions and limit potential damage by untrusted code can be classified into 3 categories, which are: "client-side setup", "server-side setup", and "a mix of both".

\subsection{Client-Side setup}
By executing the code on client side, the execution environment is ran client-side. This means that all potential damage is limited to the client only. If the code is malicious, their own page is broken and not the server.

Since browsers dont natively allow python code to be ran, a Python to WebAssembly interpreter is used. WebAssembly is a low level, byte code instruction set that is run directly from most browsers[22]. With a Python-to-WebAssembly interpreter, user written Python can be interpreted and compiled to byte code directly and allows for an increasingly scalable solution. However, WebAssembly is known to be unblockable and disables synchronous input. An option that was under consideration to use for the project was PyodideU, as it allows synchronous user input but it lacks documentation. Another option to run python in the browser was skulpt, but it runs on Python 2.x and not Python 3. So Pyodide, a port of CPython to WebAssembly, was used as it is easy to implement was able to be run client-side and has a large, active community. Pyodide also includes many general-purpose Python libraries, such as NumPy, Pandas, and regex, which can be used to run user code.

To ensure security, the Python execution environment also needs to be restricted so that it can't load or modify other local data files. Sensitive attributes must be hidden outside the Python execution environment where the code will be ran. This means that any execution output or execution process should be only able to communicate 1 way as shown in figure \ref{fig:execution layout}, and the code should not be able to access anything beyond its scope.

\begin{figure}[H]
    \centering
    \includegraphics[width=0.8\textwidth]{images/execution_layout.png}
    \caption{Execution layout}
    \label{fig:execution layout}
\end{figure}

The client-side setup brings many benefits and disadvantages. If the client is disconnected from the server after loading the web app, the code can still be run locally on the machine as it has serverless program execution. Since the code is also run client-side, It can help reduce server load as no code is sent over and run locally on the client's machine. The client's machine is responsible for all code execution and therefore the execution speed is also reliant on it, getting rid of latency issues and making it more interactive. However, due to it running on the client's machine, there can also be resource constraints, slow execution, and crashing in the worst case.

\subsection{Server-side setup}
A server-based setup is the more popular option for executing untrusted Python code and many such tools have been developed for it. This involves the code being sent from the client side to the server which then can be executed. Similar to methods employed in some CTFs (Capture The Flag) and other online tools, this involves the use of sandboxes. According to \cite{stephens2024sandbox}, Sandboxing is a technique for enforcing security policies on untrusted guest applications in a secure environment (i.e., “sandbox”) to eliminate risk to a host system. In this section, it's function is to securely execute untrusted application on host systems (CITE THIS). This execution happens usually in an isolated container or a seperate process with limited resources and various filters, such as system calls and network traffic etc.

\subsubsection{Docker containers}
Docker pools can also be used, such as with Microsoft Azure Cloud Shell. Security features of the operating system, e.g., seccomp/capabilities/namespaces in Linux, and Packages such as Code Jail (https://github.com/openedx/codejail) can be used to help aid the execution of untrusted code. The setup is so that the sandbox where user interaction happens is placed outside of where the main server process is. 


% Similar to client-side setup, potential benefits and disadvantages depend on how the system is designed. If one were to use docker pools (\\cite{docker_criu}) to contain code execution, there would be a significantly increased server load, as virtual machines need to be booted up and served for each connection, as compared to executing code in a separate process in the server. However, docker pools have the advantage of limiting potential damage to only the virtual machine, while if there is a security flaw in a separate process, the entire server is also affected. This setup also allows the server to log, monitor, rollbacks, and cache results from user code.

% the environment is set up in such a way that it can only access the resources and perform the actions that are allowed by the game/level designer. The sandboxed execution environment should also be able to monitor the user code and terminate it if it tries to perform any malicious actions

% 


\subsubsection{Transpilation to Javascript}
Source-to-source compilers, also known as transcompiler, transpiler or cross-compilers is a set of tools that take the source code developed in a high-level programming language (source language), and after a transpilation process, it generates a translated source code written in another programming language (target language), that is syntactically equivalent. Puder et al.[4], says that cross-compilation itself provides only the tools to translate from one language to another, but to make it usable, it is necessary to offer a library to provide the building blocks and there currently exists no such libraries to transpile the python ecosystem. Additionally, some of the most prominent and performant Python to Javascript transpilers consistently have issues with blocking synchronous user input. This introduces lots of bugs and causes crashing the user's browser, which was a huge problem during testing and implementation as it was also hard to debug.





\subsection{A mix of both}
refer to codecombat aetherjs and stuff

\subsection{Conclusion}
Both methods have their own pros and cons, and the choice of which method to use depends on the use case and the resources available. The client-side setup is more suitable for offline functionality and reducing server load, while the server-side setup is more suitable for monitoring, logging, and security. More research will be done to figure out the best method to use for the project.
		% 4. ...
\newpage

\thispagestyle{myPageStyle}
\section{Setup of code execution}
Python code submitted by the user should be executed to achieve some type of objective with constraints so that the objective can be achieved by the game/level designer's intended methods. Many unintended actions can be caused if user code is not run securely such as the modification of the intended objective, falsification of high scores, crashing the backend, or in the worst case, bypassing the game and changing system files and databases.

According to research, most methods to prevent malicious actions and limit potential damage by untrusted code can be classified into 2 categories: client-side setup and server-side setup. In this section, we will discuss the pros and cons of each method and how they can be implemented to ensure the security of the system.

\subsection{Client-Side setup}
By executing the code on client side with transpilation, the execution environment is ran client-side. This means that all potential damage is limited to the client only. If the code is malicious, their own page is broken and not the server.

To ensure security, the Python execution environment also needs to be restricted so that it can't load or modify other local data files. Sensitive attributes must be hidden outside the Python execution environment where the code will be ran. This means that any execution output should be only able to communicate 1 way as shown in figure \ref{fig:execution layout}, and the execution process should not be able to access anything beyond its scope, such as local files, backend and database.

\begin{figure}[H]
    \centering
    \includegraphics[width=0.8\textwidth]{images/execution_layout.png}
    \caption{Execution layout}
    \label{fig:execution layout}
\end{figure}


\subsubsection{Overview}
The client-side setup brings many benefits and disadvantages. If the client is disconnected from the server after loading the web app, the code can still be run locally on the machine as it has serverless program execution. Since the code is also run client-side, It can help reduce server load as no code is sent over and run locally on the client's machine. The client's machine is responsible for all code execution and therefore the execution speed is also reliant on it, getting rid of latency issues and making it more interactive. However, due to it running on the client's machine, there can also be resource constraints, slow execution, and crashing in the worst case. Any malicious intent will also just end up in the attacker breaking their own browser. Any high scores uploaded to the server can also be falsified due to the client-side execution and the server having no idea how the code was executed.

\subsection{Server-side setup}
A server-based setup is the more popular option for executing untrusted Python code and many such tools have been developed for it. This involves the code being sent from the client side to the backend which then can be executed, this usually involves the use of sandboxes. According to \cite{stephens2024sandbox}, Sandboxing is a technique for enforcing security policies on untrusted guest applications in a secure environment (i.e., “sandbox”) to eliminate risk to a host system. In this section, it's function is to securely execute untrusted application on host systems (CITE THIS). This execution happens usually in an isolated container or a seperate process with limited resources and various filters, such as system calls and network traffic etc.

Similar to the client-side setup, any execution output should only be sent without any influence to the backend database. Any execution processes should only happen within the isolated sandbox.

\subsubsection{Docker containers and Virtual Machines}
A Docker container is a lightweight, standalone, executable package of software that includes everything needed to run an application, not to be confused with Virtual machines (VMs). Similar to methods employed in some CTFs (Capture The Flag), and other online tools used to assess code(CITE THIS), Docker containers can also be used to sandbox and execute code isolatedly. The container can either be ran on a VM or on the same server as the backend.

Virtual Machines are also used to sandbox and execute code, but they are slower to boot up and can be expensive to run. They also require more resources than docker containers and can be harder to manage. However, they can be more secure than docker containers if they are isolated from the main server. These VMs can either be hosted on the same OS the server is ran on or hosted on the cloud with popular services such as Microsoft Azure and AWS EC2 instances.

The architecture of a server-side setup with docker containers would look like in figure \ref{fig:docker container setup}. It begins with the frontend sending code to the backend to run it through the container. The backend sends the code to the container along with any relevant information, such as the current level. This enables the container to run any pre defined logic and checks based on relevant information. The container then does validation and execution of the code. The container will be running a seperate server that can handle code execution and sends the standard output back to the backend, which then sends the results back to the client. 

\begin{figure}
    \centering
    \includegraphics[width=0.8\textwidth]{images/server_execution_layout.png}
    \caption{Docker container setup}
    \label{fig:docker container setup}
\end{figure}

To fire up and setup the container, a DockerFile which contains instructions on how to build the Docker image is used, this usually instructs what type of dependencies to install, setting environment variables and copying files etc. The DockerFile is then built into a Docker image using the 'docker build' command. The image is then used to create a docker container using the 'docker run' command, which is a shortcute of "docker create" and "docker start". A tool that can be used to speed up service running process is docker-compose. However, using a “dockercompose.yml” file, a container (or more than one) can be run with a single command (docker-compose up), in the background (using “-d” option) and the image can be built each time before container starts again.

To setup a virtual machine, a filesystem is needed, and a VM is needed to be spun up. Popular VM options include VirtualBox, VMWare, and Hyper-V. VMs can also run almost any operating software, the most common being some form of Linux distro such as Ubuntu. VM security is also treated very seriously and there have also been a lot more history and security research done about VMs compared to Docker containers, such as seccomp which was initially developed as a way to restrict system calls for Linux kernels. VMs also have the advantage of customization and OS configuration to install OS specific packages such as CodeJail to help aid the execution of untrusted code.

To gurantee absolute isolation and safety, both VMs and Docker containers can be used to construct the sandbox execution environment by hosting the Docker container on a cloud hosted VM that is seperate from the main backend.

\subsubsection{Overview}
Similar to client-side setup, potential benefits and disadvantages depend on how the system is designed. If one were to use docker containers to contain code execution, there would be a significantly increased server load, as virtual machines need to be booted up and served for each connection, as compared to executing code in a separate process in the server. However, docker containers have the advantage of limiting potential damage to only the OS it is hosted on. This setup also allows the server to log, monitor, rollbacks, and cache results from user code.

Another potential addon to the server-side setup is to by applying client-side code enforcement before the code is sent to the backend. In this approach, the client-side environment is used to enforce initial constraints and perform preliminary checks on the user-submitted code. This can include syntax validation, basic security checks, and ensuring that the code adheres to predefined rules. By doing this on the client side, we can quickly filter out obviously malicious or incorrect code before it reaches the server and reject it before it is sent, reducing the server load and improving responsiveness.

\subsection{Conclusion}
Both methods have their own pros and cons, and the choice of which method to use depends on the use case and the resources available. The client-side setup is more suitable for offline functionality and reducing server load, while the server-side setup is more suitable for monitoring, logging, and security.

			% 5. ...
\newpage

\thispagestyle{myPageStyle}
\section{Safeguarding against malicious code output}
If the setup for code execution is good, no code should be able to access anything beyond its scope, such as local files, backend, and database. The execution output should only be able to communicate one way and execution can't load or modify other local data files beyond its scope. However, code execution is not the only thing that can be malicious. Post code execution can also change registers, memory, and other processes. This is why it is important to have a good setup for code execution and also to have a good setup for monitoring and logging.			% 6. Study design and execution
\newpage

\thispagestyle{myPageStyle}
\section{Safeguarding against malicious code output}
If the setup for code execution is good, no code should be able to access anything beyond its scope, such as local files, backend, and database. The execution output should only be able to communicate one way and execution can't load or modify other local data files beyond its scope. However, code execution is not the only thing that can be malicious. Post code execution can also change registers, memory, and other processes. This is why it is important to have a good setup for code execution and also to have a good setup for monitoring and logging.


\section{Conclusion}
Researching Python-based execution environments is a research directive that warrants more attention as the trend of educational games continues, such as Google’s Blockly \cite{google_blockly} and China’s Yuanfudao's Homework Help \cite{xiaoyuan_kousuan}.
The future steps of continuing to research Python-based execution environments are to look at key considerations such as the performance of both methods, what makes them secure, and possible alternative implementation methods. I also need to elaborate and clear up the process workflow such as detection/validation, cleaning/sanitization, submission, and evaluation phase. Lastly, I would also need to iterate on the design for modularity and scalability. The goal is to ensure a safe and secure working environment for what is needed to run such a service.			% 7. Evaluation and Results
\newpage

% \thispagestyle{myPageStyle}
% % Kapitel 9 - Fazit

\section{Conclusion}
Researching Python-based execution environments is a research directive that warrants more attention as the trend of educational games continues, such as Google’s Blockly \cite{google_blockly} and China’s Yuanfudao's Homework Help \cite{xiaoyuan_kousuan}.
The future steps of continuing to research Python-based execution environments are to look at key considerations such as the performance of both methods, what makes them secure, and possible alternative implementation methods. I also need to elaborate and clear up the process workflow such as detection/validation, cleaning/sanitization, submission, and evaluation phase. Lastly, I would also need to iterate on the design for modularity and scalability. The goal is to ensure a safe and secure working environment for what is needed to run such a service.

% \section{Fazit}
% Diese Arbeit befasste sich grundsätzlich mit der Frage, welche Anforderungen an ... gestellt werden.
% Der Lösungsvorschlag war...
% Benutzerstudien haben gezeigt, dass es einen signifikanten Unterschied... gibt. Diese Ergebnisse motivieren, um ...

% \subsection{Ausblick} 
% In dieser Arbeit haben wir uns mit ... beschäftigt. Durch zwei Benutzerstudien wurde festgestellt, dass sich die Aufteilung und Positionierung der Informationen innerhalb der Anzeige der beiden Gruppen aufgrund bla bla ändert. Das entwickelte Konzept ist zwar bla bla, müsste aber für eine Wiederholung der Studien bla bla zur Erhebung quantitativer Daten wie folgt angepasst werden:

% \begin{itemize}
% \item xxx
% \item yyy
% \item xzz
% \end{itemize}

% Des Weiteren wurde festgestellt, dass bla bla. Auch das müssten zukünftige Varianten besser berücksichtigen, z.\,B., indem sie bla bla. Für Studien sollte zudem eine neutrale(re) Umgebung gewählt werden. Somit sollte sichergestellt werden können, dass beispielsweise ein möglicher Bias der Marke des Fahrzeugs sich nicht auf das HMI-Konzept auswirkt...

% \subsection{Einschränkungen}
% Hinsichtlich Einschränkungen, die eine breite Anwendung der Ergebnisse verhindern, sind zwei getrennte Bereiche zu betrachten. Zum einen die Evaluation, welche sich nur auf den ersten Teil der Arbeit bezieht, zum anderen das ausgearbeitete Konzept für die Anzeige, das auf dem ersten Teil der Arbeit aufbaut. \\ [-2.5em]

% \paragraph{Evaluation} 
% Bla bla.

% \paragraph{Konzept}
% Bla bla. Das wurde etwa in beschrieben...

% %\input{chap9x} %chap9_futurework_limitations}			% 8. Discussion, Deriving concrete action recommendations
% \newpage

% \thispagestyle{myPageStyle}	% 9. Conclusion, Future work, Limitations
% % Kapitel 9 - Fazit
\section{Fazit}
Diese Arbeit befasste sich grundsätzlich mit der Frage, welche Anforderungen an ... gestellt werden.
Der Lösungsvorschlag war...
Benutzerstudien haben gezeigt, dass es einen signifikanten Unterschied... gibt. Diese Ergebnisse motivieren, um ...

\subsection{Ausblick} 
In dieser Arbeit haben wir uns mit ... beschäftigt. Durch zwei Benutzerstudien wurde festgestellt, dass sich die Aufteilung und Positionierung der Informationen innerhalb der Anzeige der beiden Gruppen aufgrund bla bla ändert. Das entwickelte Konzept ist zwar bla bla, müsste aber für eine Wiederholung der Studien bla bla zur Erhebung quantitativer Daten wie folgt angepasst werden:

\begin{itemize}
\item xxx
\item yyy
\item xzz
\end{itemize}

Des Weiteren wurde festgestellt, dass bla bla. Auch das müssten zukünftige Varianten besser berücksichtigen, z.\,B., indem sie bla bla. Für Studien sollte zudem eine neutrale(re) Umgebung gewählt werden. Somit sollte sichergestellt werden können, dass beispielsweise ein möglicher Bias der Marke des Fahrzeugs sich nicht auf das HMI-Konzept auswirkt...

\subsection{Einschränkungen}
Hinsichtlich Einschränkungen, die eine breite Anwendung der Ergebnisse verhindern, sind zwei getrennte Bereiche zu betrachten. Zum einen die Evaluation, welche sich nur auf den ersten Teil der Arbeit bezieht, zum anderen das ausgearbeitete Konzept für die Anzeige, das auf dem ersten Teil der Arbeit aufbaut. \\ [-2.5em]

\paragraph{Evaluation} 
Bla bla.

\paragraph{Konzept}
Bla bla. Das wurde etwa in beschrieben...

%\input{chap9x} %chap9_futurework_limitations}
% \newpage	

	%\shorthandon{"}
	
%--------------------------------------------------------------------------------
%-----Anhang---------------------------------------------------------------------
%--------------------------------------------------------------------------------
	
	\pagenumbering{Roman} 					% Römische Nummerierung der Kapitel/Roman page numbering
	\setcounter{page}{6} 						% Beginn bei Seitenzahl X (hier: 6) um bei oberer Nummerierung aufzuschließen/Adapt page numbering
	
	%Glossar/Glossary
	\thispagestyle{myPageStyle}
	\glssetwidest{A D A S} 						% gleicher Abstand zur 2. Spalte (längstes Wort)					
	\setglossarystyle{alttree}																	
	%\printglossary[title=Abkürzungsverzeichnis,toctitle=Abkürzungsverzeichnis] 	% Rename for German thesis
	\cleardoublepage
		
	
	%Literaturliste/Literature references
	\thispagestyle{myPageStyle}
	\bibliographystyle{plainnat}
	%\bibliographystyle{abbrv}% changed abbrvdin to abbrv % DIN-Norm für Literaturdarstellung  plaindin 
	%\bibliographystyle{abbrvnat}
	\setcitestyle{authoryear, open={(}, close={)}}
	
	\renewcommand{\refname}{Literature references} % Remove for German thesis
	\bibliography{literature}					% Pfad und Datei der Literaturdatenbank/Path and file name of literature references
	\cleardoublepage	
	
	%Anhänge/Appendices
	%\thispagestyle{myPageStyle} 
	%%----------Anhang/Appendices--------------------------------------------------------------

\appendix
\section{Anhang}

\subsection[]{Onlinefragebogen} \label{sec:a1} 
%\vspace{2em}
%\includepdf[pages={2-3}]{images/fragebogen/questionnaire2}

\subsection[]{Rohdaten des Onlinefragebogens für die statistische Auswertung}

\subsection[]{Auswertung der Evaluation hinsichtlich...}

\subsection[]{Datenträger/Data carrier}
	%\cleardoublepage
	
%----------------------------------------------------------------------------------
%----------------DOKUMENTENENDE - END OF DOCUMENT----------------------------------
%----------------------------------------------------------------------------------
	
\end{document}